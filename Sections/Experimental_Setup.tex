
\section{Experimental Setup} 

% (fold)
\label{sec:experimental_setup} The two main decay processes in this experiment are gamma and beta. Since we would only like to study beta decay, we have to eliminate the detection of gamma particles. Gamma particles penetrate materials much further than beta particles. By choosing the scintillator in configuration 2, we have a smaller distance over which to absorb and detect radiation. Therefore, this scintillator will detect the beta particles and allow the majority of gamma particles to pass through undetected. It takes more material to stop gamma particles than electrons because gamma particles have a lower probability to interact with materials.  This allows them to travel large distances without any interactions(or loss in energy).

To calibrate the scintillator, we used the discrete energy value given from Cesium's internal conversion, $E=624.3~\text{eV}$. Table~\ref{tab:calibrationNumbers} shows the values we used to calibrate our system from channel numbers to energy values.  The naive assumption was made that the zero channel corresponds to zero energy because we did not have any other known values like the discrete energy given from $^{137}\text{Cs}$. This possible source of error is addressed in Section~\ref{sec:discussion_conclusion}, Discussion/Conclusion.
\begin{table}
	[tbp] 
	\begin{center}
		\begin{tabular}
{ll} \toprule Channel Number & Corresponding Energy\\
                \midrule $0$ &        $0~\text{eV}$\\
                      $1838$ &    $624.3~\text{keV}$\\
\bottomrule 
		\end{tabular}
	\end{center}
	\caption{The values we used to calibrate our system in order to relate channel numbers to energy values.} \label{tab:calibrationNumbers} 
\end{table}%(tab:calibrationNumbers)


% section experimental_setup (end)