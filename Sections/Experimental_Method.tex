%!TEX root = /Users/kevin/SkyDrive/KTH Work/LaTeX Reports/Atomic Nucleus Revised/Atomic Nucleus.tex

\section{Experimental Method} % (fold)
\label{sec:experimental_method}

The description below describes our procedure after calibration of the equipment ($n$ goes from $ 0 \to 6$ for aluminum and $0 \to 5 \text{~and~} 7$, for plastic). For plastic, $n=6$ was skipped only because time was limited.   
% In hindsight, we should have followed some criteria to determine whether or not a peak is distinguishable, instead of ``eyeballing'' it.  
\begin{enumerate}
	\item Place $n$ sheets of absorbing material (aluminum or plastic) between $^{137}\text{Cs}$ and the detector.
	\item Run PC program Multi Channel Analyser Tukan to collect a set number of counts for each measurement. The measurement with aluminum did not use this criteria, only plastic.
	\item Estimate peak center and error by taking the average of the left and right peak estimations.
	\item Obtain list of energies and thicknesses of absorbing material.
	\item Estimate the thickness needed to stop beta particles ($E\approx0$) using a linear approximation.
	\item Estimate and plot $R(E)=\text{k}E^{\text{B}}$, where k and B are constants and $R(E)$ is the length of material needed to stop an electron of energy $E$.
    Estimation of the stopping distance is done by assuming a linear relationship between number of sheets and electron energy.  As more sheets are added, the energies of the collected electrons become smaller and less frequent as they are being scattered from collisions with the absorbing material.
	\item Compare results with similar experiments.\cite{RevModPhys.24.28,nistData}
\end{enumerate}%(Steps for experiment)

% section experimental_method (end)